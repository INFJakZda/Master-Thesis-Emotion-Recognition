\chapter{Wstęp}

Emocje międzyludzkie są podstawą codziennych interakcji z innymi osobami, a badania naukowców pokazują, że emocje są zjawiskiem uniwersalnym dla ludzi bez względu na pochodzenie. Jednak wpływy kulturowe jak i interpersonalne odgrywają kluczową rolę w identyfikacji konkretnych nawet podstawowych emocji, takich jak radość, miłość, gniew, strach i złość. Im bardziej wyszczególnione są etykiety emocji, tym trudniej jest wykryć tę właściwą. System rozpoznawania emocji może okazać się przydatny do wzajemnego zrozumienia między osobami, poprzez dostarczenie niewykrytego sygnału emocji. 

Doskonałym przykładem są emocje występujące w mediach społecznościowych, które w niektórych sytuacjach są bardzo wyraziste, lecz bywają też niejednoznaczne i przy tym mogą być wyrażane w wulgarny sposób. System wykrywania emocji w obszarze dialogów między ludźmi w postaci rozmowy na forum internetowym lub komentarzy pod postem może okazać się bardzo pomocny w poprawieniu bezpieczeństwa w Internecie dla ludzi młodych i dzieci. Przykładów zastosowań jest mnóstwo, kilka z nich to filtrowanie treści, blokada słów wulgarnych i obraźliwych oraz zdań z ukrytym podtekstem. Jest to także rekomendacja treści, grupowanie tekstu o podobnym znaczeniu, czy nawet badanie rynku. Można by wykorzystać taki system do zbadania większej liczby komentarzy, np. w przypadku koncernów samochodowych, czy firm produkujących elektronikę jakie emocje przewyższają w komentarzach pod wyświetlanymi reklamami w mediach społecznościowych. Czy jest to zachwyt, zadowolenie, czy rozczarowanie. Działania te mogły by odpowiedzieć na pytanie czy wydany właśnie przez nich produkt przyjmie się na rynku oraz co warto by poprawić. W takich przypadkach etykietowanie komentarzy przez człowieka mogłoby okazać się trudne do wykonania i niejednoznaczne oraz bardzo kosztowne i czasochłonne.

\section{Cel i zakres pracy}

Głównym celem pracy jest opracowanie modelu uczenia maszynowego opartego na głębokich sieciach neuronowych (klasyfikatora wieloklasowego) w celu detekcji emocji w tekście postów z dialogów z mediów społecznościowych. W ramach tego celu niezbędne będzie zapoznanie się z literaturą dotyczącą przetwarzania języka naturalnego i dostępnymi frameworkami do uczenia głębokiego, przeprowadzenie analizy eksploracyjnej wybranych zbiorów danych, wstępne przetworzenie tych danych po czym nastąpi budowa i ewaluacja kilku modeli o różnych architekturach w celu przeprowadzenia analizy porównawczej.


\section{Struktura pracy}

Struktura pracy jest następująca. Rozdział 2 przedstawia podstawy teoretyczne wymagane do zrozumienia dalszych etapów pracy wraz z przeglądem literatury. Rozdział 3 jest poświęcony analizie eksploracyjnej wybranych zbiorów danych. Rozdział 4 zawiera techniki przetwarzania wstępnego zbiorów danych. Rozdział 5 ukazuje budowę modeli a rozdział 6 ich ewaluację. Rozdział 7 zawiera podsumowanie pracy.
