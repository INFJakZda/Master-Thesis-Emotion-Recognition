\chapter{Podstawy teoretyczne}

Rozpoznawanie emocji w dialogach koncentruje się na wydobyciu emocji przekazanej w rozmowie pomiędzy co najmniej dwoma rozmówcami. Problem ten stawia bardzo dużo wyzwań, takich jak obecność sarkazmu w rozmowie, przesunięcie emocji do kolejnych wypowiedzi tego samego rozmówcy oraz uchwycenie szerszego kontekstu pomiędzy wypowiedziami różnych rozmówców. Dużym plusem w tej dziedzinie jest bardzo dobra dostępność do danych, które pochodzą z platform społecznościowych takich jak Facebook, Youtube, Reddit, Twitter \cite{poria2019emotion}. Poprzez łatwą dostępność do danych rozpoznawanie emocji w rozmowie staje się coraz bardziej popularne, a trudność tego problemu stwarza coraz to bardziej odległe granice co sprowadza się do wysokiego zainteresowania tą dziedziną przetwarzania języka naturalnego (ang. \textit{NLP}).

Bardzo ważnym elementem w rozpoznawaniu emocji jest możliwość zrozumienia danego przekazu w kontekście, od którego może zależeć rodzaj emocji. Szczególnie trudnym przypadkiem jest zrozumienie i zapamiętanie kontekstu w konwersacji, co jak pokazuje \cite{zhong2019knowledgeenriched}, może okazać się kluczowym czynnikiem skuteczności rozpoznawania emocji. Do uzyskania satysfakcjonujących wyników nie wystarczają tradycyjne metody uczenia maszynowego lub najbardziej podstawowe architektury sieci neuronowych. Modele te wykorzystują zaawansowane techniki architektury transformera \cite{vaswani2017attention} które korzystają z podejście mającego na celu poprawę modelowania sekwencja do sekwencji (ang. \textit{Seq2Seq}) poprzez samoobserwację (ang. \textit{self-attention}) i kodowanie pozycji (ang. \textit{positional encoding}).

TODO: opisz coś o różnych modelach emocji

\cite{howard2018universal}

\cite{devlin2018bert}

\cite{brochier2019global}