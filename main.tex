%%%%%%%%%%%%%%%%%%%%%%%%%%%%%%%%%%%%%%%%%%%%%%%%%%
%% Bachelor's & Master's Thesis Template        %%
%% Copyleft by Dawid Weiss & Marta Szachniuk    %%
%% Faculty of Computing and Telecommunication   %%
%% Poznan University of Technology, 2020        %%
%%%%%%%%%%%%%%%%%%%%%%%%%%%%%%%%%%%%%%%%%%%%%%%%%%


% Szkielet dla pracy licencjackiej pisanej w języku polskim.

\documentclass[polish,a4paper,oneside]{ppfcmthesis}


\usepackage[utf8]{inputenc}
\usepackage[OT4]{fontenc}

%--------------------------------------
% Strona tytułowa
%--------------------------------------

% Autorzy pracy, jeśli jest ich więcej niż jeden
% wstaw między nimi separator \and
\author{%
   inż. Jakub Zdanowski \album{127239}
}
\authortitle{}                                % Do not change.

\title{Rozpoznawanie emocji w mediach społecznościowych z wykorzystaniem głębokich sieci neuronowych}

% Your supervisor comes here.
\ppsupervisor{~dr hab.~inż.~Agnieszka Ławrynowicz} 

% Year of final submission (not graduation!)
\ppyear{2020}                                 


\begin{document}

% Front matter starts here
\frontmatter\pagestyle{empty}%
\maketitle\cleardoublepage%

%--------------------------------------
% Miejsce na kartę pracy dyplomowej
%--------------------------------------

\thispagestyle{empty}\vspace*{\fill}%
\begin{center}Tutaj będzie karta pracy dyplomowej;\\oryginał wstawiamy do wersji dla archiwum PP, w pozostałych kopiach wstawiamy ksero.\end{center}%
\vfill\cleardoublepage%

%--------------------------------------
% Spis treści
%--------------------------------------

\pagenumbering{Roman}\pagestyle{ppfcmthesis}%
\tableofcontents* 
\cleardoublepage % Zaczynamy od nieparzystej strony

%--------------------------------------
% Rozdziały
%--------------------------------------

%Najwygodniej jeśli każdy rozdział znajduje się w oddzielnym pliku
\mainmatter%
\chapter{Wstęp}

Emocje międzyludzkie są podstawą codziennych interakcji z innymi osobami, a badania naukowców pokazują, że emocje są zjawiskiem uniwersalnym dla ludzi bez względu na pochodzenie. Jednak wpływy kulturowe jak i interpersonalne odgrywają kluczową rolę w identyfikacji konkretnych nawet podstawowych emocji, takich jak radość, miłość, gniew, strach i złość. Im bardziej wyszczególnione są etykiety emocji, tym trudniej jest wykryć tę właściwą. System rozpoznawania emocji może okazać się przydatny do wzajemnego zrozumienia między osobami, poprzez dostarczenie niewykrytego sygnału emocji. 

Doskonałym przykładem są emocje występujące w mediach społecznościowych, które w niektórych sytuacjach są bardzo wyraziste, lecz bywają też niejednoznaczne i przy tym mogą być wyrażane w wulgarny sposób. System wykrywania emocji w obszarze dialogów między ludźmi w postaci rozmowy na forum internetowym lub komentarzy pod postem może okazać się bardzo pomocny w poprawieniu bezpieczeństwa w Internecie dla ludzi młodych i dzieci. Przykładów zastosowań jest mnóstwo, kilka z nich to filtrowanie treści, blokada słów wulgarnych i obraźliwych oraz zdań z ukrytym podtekstem. Jest to także rekomendacja treści, grupowanie tekstu o podobnym znaczeniu, czy nawet badanie rynku. Można by wykorzystać taki system do zbadania większej liczby komentarzy, np. w przypadku koncernów samochodowych, czy firm produkujących elektronikę jakie emocje przewyższają w komentarzach pod wyświetlanymi reklamami w mediach społecznościowych. Czy jest to zachwyt, zadowolenie, czy rozczarowanie. Działania te mogły by odpowiedzieć na pytanie czy wydany właśnie przez nich produkt przyjmie się na rynku oraz co warto by poprawić. W takich przypadkach etykietowanie komentarzy przez człowieka mogłoby okazać się trudne do wykonania i niejednoznaczne oraz bardzo kosztowne i czasochłonne.

\section{Cel i zakres pracy}

Głównym celem pracy jest opracowanie modelu uczenia maszynowego opartego na głębokich sieciach neuronowych (klasyfikatora wieloklasowego) w celu detekcji emocji w tekście postów z dialogów z mediów społecznościowych. W ramach tego celu niezbędne będzie zapoznanie się z literaturą dotyczącą przetwarzania języka naturalnego i dostępnymi frameworkami do uczenia głębokiego, przeprowadzenie analizy eksploracyjnej wybranych zbiorów danych, wstępne przetworzenie tych danych po czym nastąpi budowa i ewaluacja kilku modeli o różnych architekturach w celu przeprowadzenia analizy porównawczej.

\section{Struktura pracy}

Struktura pracy jest następująca. Rozdział 2 przedstawia podstawy teoretyczne wymagane do zrozumienia dalszych etapów pracy wraz z przeglądem literatury. Rozdział 3 jest poświęcony analizie eksploracyjnej wybranych zbiorów danych. Rozdział 4 zawiera techniki przetwarzania wstępnego zbiorów danych. Rozdział 5 ukazuje budowę modeli a rozdział 6 ich ewaluację. Rozdział 7 zawiera podsumowanie pracy.

\chapter{Podstawy teoretyczne}

Rozpoznawanie emocji w dialogach koncentruje się na wydobyciu emocji przekazanej w rozmowie pomiędzy co najmniej dwoma rozmówcami. Problem ten stawia bardzo dużo wyzwań, takich jak obecność sarkazmu w rozmowie, przesunięcie emocji do kolejnych wypowiedzi tego samego rozmówcy oraz uchwycenie szerszego kontekstu pomiędzy wypowiedziami różnych rozmówców. Dużym plusem w tej dziedzinie jest bardzo dobra dostępność do danych, które pochodzą z platform społecznościowych takich jak Facebook, Youtube, Reddit, Twitter \cite{poria2019emotion}. Poprzez łatwą dostępność do danych rozpoznawanie emocji w rozmowie staje się coraz bardziej popularne, a trudność tego problemu stwarza coraz to bardziej odległe granice co sprowadza się do wysokiego zainteresowania tą dziedziną przetwarzania języka naturalnego (ang. \textit{NLP}).

Bardzo ważnym elementem w rozpoznawaniu emocji jest możliwość zrozumienia danego przekazu w kontekście, od którego może zależeć rodzaj emocji. Szczególnie trudnym przypadkiem jest zrozumienie i zapamiętanie kontekstu w konwersacji, co jak pokazuje \cite{zhong2019knowledgeenriched}, może okazać się kluczowym czynnikiem skuteczności rozpoznawania emocji. Do uzyskania satysfakcjonujących wyników nie wystarczają tradycyjne metody uczenia maszynowego lub najbardziej podstawowe architektury sieci neuronowych. Modele te wykorzystują zaawansowane techniki architektury transformera \cite{vaswani2017attention} które korzystają z podejście mającego na celu poprawę modelowania sekwencja do sekwencji (ang. \textit{Seq2Seq}) poprzez samoobserwację (ang. \textit{self-attention}) i kodowanie pozycji (ang. \textit{positional encoding}).

TODO: opisz coś o różnych modelach emocji

\cite{howard2018universal}

\cite{devlin2018bert}

\cite{brochier2019global}
\chapter{Analiza eksploracyjna}

\section{Zbiór danych - EmoContext}

Zbiór danych został udostępniony przez zespół z Międzynarodowych Warsztatów Ewaluacji Semantycznej (ang. \textit{SemEval-2019 International Workshop on Semantic Evaluation}) jako jedno z zadań konkursowych o nazwie EmoContext\footnote{\url{https://www.humanizing-ai.com/emocontext.html}} (ang. \textit{Contextual Emotion Detection in Text}).  Celem jest odkrycie prawidłowej etykiety emocji dla danego dialogu, składającego się z trzech wypowiedzi. W zadaniu tym użyty jest uproszczony model emocji zaproponowany przez Paula Ekmana \cite{ekman1993facial}, składający się z czterach klas: \textit{Happy}, \textit{Sad}, \textit{Angry} oraz \textit{Others}. Przykładowe dialogi oraz odpowiadające im etykiety emocji pochądzące ze zbioru treningowego zaprezentowane są na rysunku \ref{rys:examples_semeval}.

\begin{figure}[h]
\centering\includegraphics[width=12cm]{figures/examples_semeval.png}
\fcmfcaption{Przykładowe dialogi w danych z EmoContext wraz z etykietą emocji.}\label{rys:examples_semeval}
\end{figure}

Przez organizatorów konkursu udostępnione zostały następujące zbiory do ewaluacji własnych modeli: zbiór treningowy (\textit{train}), zbiór walidacyjny (\textit{dev}) oraz zbiór testowy (\textit{test}). Każdy z tych zbiorów zawiera tą samą strukturę danych, każdy przykład składa się z identyfikatora (\textit{Id}), trzech wypowiedzi w dialogu oraz etykiety emocji. Szczegółowe informacje na temat tych zbiorów ukazuje tabela \ref{tab:tabela_emocontext}.

\begin{table}[ht]
\fcmtcaption{Tabela prezentująca informacje o zbiorach w danych z EmoContext.}\label{tab:tabela_emocontext}
\centering\footnotesize%
\begin{tabular}{c c c}
\toprule
zbiór & liczba przykładów & najczęstsza klasa \\
\midrule
train   & 30160 & \textit{Others} 50\% \\
dev   & 2755 & \textit{Others} 84\% \\
test   & 5509 & \textit{Others} 84\% \\
\bottomrule
\end{tabular}
\end{table}

Aby lepiej zrozumieć dane treningowe przeprowadzone zostały podstawowe analizy eksploracyjne tego zbioru. Jedną z nich jest rozkład długości całego dialogu zaprezentowany na rysunku \ref{rys:rozklad_dl_semeval}. Widzimy na nim że przeważające dialogi są dosyć krótkie, średnio 62 znaki, co może utrudnić zadanie rozpoznania właściwej emocji. Zauważyć można także występujący długi ogon w kierunku coraz dłuższych dialogów, najdłuższy z nich ma 692 znaki.

\begin{figure}[t]
\centering\includegraphics[width=12cm]{figures/rozklad_dl_semeval.png}
\fcmfcaption{Rozkład długości dialogu w danych z EmoContext.}\label{rys:rozklad_dl_semeval}
\end{figure}

Kolejna analiza to histogram częstości występowania danej etykiety w zbiorze treningowym, dzięki temu można sprawdzić czy występuje niezbalansowanie klas. Rysunek \ref{rys:rozklad_liczby_klas_semeval} prezentuje dominację klasy \textit{Others} nad pozostałymi klasami, jest średnio ponad dwukrotnie liczniejszy od każdej z pozostałych etykiet. 

\begin{figure}[t]
\centering\includegraphics[width=8cm]{figures/rozklad_liczby_klas_semeval.png}
\fcmfcaption{Niezbalansowany rozkład etykiet emocji w danych z EmoContext.}\label{rys:rozklad_liczby_klas_semeval}
\end{figure}

Ostatnim elementem analizy eksploracyjnej tego zbioru jest identyfikacja najczęściej występujących \textit{N-gramów}, gdzie \textit{N} to liczba zlepek słów występujących obok siebie w dialogach. Rysunki \ref{rys:semeval_1_gram}, \ref{rys:semeval_2_gramy} oraz \ref{rys:semeval_3_gram} ukazują histogramy najczęstszych \textit{N-gramów}. Można zauważyć że najczęściej występującymi słowami są słowa należące do grupy słów nie wnoszących znaczenia (ang. \textit{stop words}), np.: słowa popularne, spójniki, przedimki, jednak często występujące słowa to także wnoszące dużo znaczenia do wypowiedzi takie jak \textit{love}, \textit{like}, \textit{good}, \textit{hate}.

\begin{figure}[t]
\centering\includegraphics[width=\textwidth]{figures/semeval_1_gram.png}
\fcmfcaption{Rozkład wystąpień najczęstszych 1-gramów (wyrazów) w danych z EmoContext.}\label{rys:semeval_1_gram}
\end{figure}

\begin{figure}[t]
\centering\includegraphics[width=\textwidth]{figures/semeval_2_gramy.png}
\fcmfcaption{Rozkład wystąpień najczęstszych 2-gramów w danych z EmoContext.}\label{rys:semeval_2_gramy}
\end{figure}

\begin{figure}[t]
\centering\includegraphics[width=\textwidth]{figures/semeval_3_gram.png}
\fcmfcaption{Rozkład wystąpień najczęstszych 3-gramów w danych z EmoContext.}\label{rys:semeval_3_gram}
\end{figure}

\chapter{Przetwarzanie danych}

\section{Wczytanie i oczyszczenie danych}

Pierwszym etapem przygotowania danych do użycia w modelu jest ich wczytanie oraz przetworzenie. Zbiór danych z EmoContext zawiera pięć kolumn (rys. \ref{rys:examples_semeval}). W pierwszej kolumnie znajduje się identyfikator, w kolumnach od 2 do 4 zawarty jest docelowy tekst który będzie użyty jako wejście do modelu, a w ostatniej kolumnie znajduje się etykieta emocji. Po wczytaniu danych następuje połączenie trzech kolumn z tekstem w jeden ciąg znaków, oddzielone znakiem specjalnym \textit{EOS} (ang. \textit{end of sentence}). Jest to niezbędna operacja która umożliwi docelowemu modelowi oddzielić trzy wypowiedzi od siebie. Następnie na tak otrzymanym ciągu znaków wykonywane są operacje usuwania powtarzających się znaków interpunkcyjnych (np. \textit{"!!!!"} zostanie zamienione na pojedynczy znak \textit{"!"}). Na końcu wykonywane jest oddzielenie znaków interpunkcyjnych od wyrazów, usunięcie powtarzających się spacji oraz zamiana wszystkich dużych liter na małe .

\section{Tokenizacja i wyrównanie}

Modele głębokie przetwarzania języka naturalnego nie operują na jawnym tekście w postaci ciągu znaków, tylko na reprezentacji w postaci liczbowej. Do reprezentacji wyrazów używane są słowniki które przechowują wszystkie znane słowa z korpusu uczącego. Do tej zamiany użyta została metoda \textit{Tokenizer} z pakietu TensorFlow. Klasa ta pozwala na wektoryzację korpusu tekstu, poprzez przekształcenie każdego wyrazu w liczbę całkowitą. Każda z tych liczb odpowiada indeksowi tego słowa w słowniku.

Po przekształceniu każdego słowa w token następuje wyrównanie długości wszystkich przykładów w zbiorze. Przykładowa liczba użyta do wyrównania (ang. \textit{padding}) to 20. Wszystkie przykłady, które mają mniejszą liczbę tokenów, wypełniane są tokenem zerowym aż do osiągnięcia długości 20, a wszystkie przykłady które mają większą liczbę tokenów są skracane od końca. W ten sposób została utworzona macierz danych która może być wykorzystana do użycia w nauce i ewaluacji modeli.

\section{Zagłębienia słów}

W przypadku głębokich sieci neuronowych z rekurencyjnymi komórkami LSTM wysoce efektywne jest użycie zagłębień słów (ang. \textit{words embeddings}) jako wejścia do pierwszej warstwy sieci neuronowej. Osadzanie słów jest przeprowadzone za pomocą mapowania tych słów na wektory liczb rzeczywistych. Reprezentacja słów jako wektory liczb jest rozszerzeniem reprezentacji \textit{jeden z N} (ang. \textit{one hot encodings}), która jest używana w celu zwiększenia wydajności modeli NLP. Jest to także możliwość użycia transferu wiedzy w postaci wstępnie wytrenowanych zagłębień słów jako reprezentacji danych wejściowych. Jednym z algorytmów, który umożliwia stworzenie takiej reprezentacji jest metoda GloVe  \cite{brochier2019global} (ang. \textit{Global Vectors for Word Representation}), stworzona przez zespół z Uniwersytetu Stanford. Udostępniona przez nich macierz umożliwia użycie tej reprezentacji, jako odwzorowanie słów na wektory liczb rzeczywistych. Słowa zmapowane w tej przestrzeni mają zachowane pewne właściwości, odległość między nimi jest powiązana z podobieństwem semantycznym. Na rysunku \ref{rys:visualize_glove} ukazane są przykładowe słowa i zachowane podobieństwo np. między słowami love i beautiful. Sposób uzyskania tej reprezentacji bazuje na współwystępowaniu danych słów w korpusie w otoczeniu które je definiuje. Główną intuicją tego modelu jest założenie, że proporcje prawdopodobieństwa współwystępowania słów mają potencjał do kodowania jakiejś formy znaczenia. Wynikiem tej metody są wektory słów które bardzo dobrze radzą sobie z zadaniami opartymi o podobieństwo, analogię, a także odkrywaniu semantyki emocjonalnej słów i wiele innych wymienionych w pakiecie \textit{word2vec} \cite{mikolov2013efficient}.

\begin{figure}[t]
\centering\includegraphics[width=\textwidth]{figures/visualize_glove.png}
\fcmfcaption{Przykładowe słowa w przestrzeni wektorowej GloVe  \cite{brochier2019global}.}\label{rys:visualize_glove}
\end{figure}


\chapter{Budowa modeli}

\section{Jednowarstwowa architektura LSTM}
\label{section:one_lstm}

Pierwszy model bazuje na zaproponowanej przez organizatorów konkursu jednowarstwowej architekturze sieci rekurencyjnej z komórkami LSTM. Zastosowanie tej budowy modelu zapewnia zdolność do uczenia się długoterminowych zależności przy jednoczesnym unikaniu długotrwałego problemu uzależnienia. 

Przygotowawczym etapem modelu jest przedstawienie wymagań dla danych, które mogą być przetwarzane w warstwie wejściowej (ang. \textit{Input Layer}). Każdy przykład jest przetworzony zgodnie z opisem w rozdziale \ref{chapter:przetwarzanie_danych}. Jednym z ostatnich etapów przetwarzania jest wyrównanie wszystkich przykładów do równej długości w tym przypadku 200, zgodnie z rysunkiem \ref{rys:lstm_one_graph} przedstawiającym budowę jednowarstwowej architektury LSTM.

\begin{figure}[t]
\centering\includegraphics[width=6cm]{figures/reports/lstm_one_graph.png}
\fcmfcaption{Graf przedstawiający budowę jednowarstwowej architektury LSTM.}\label{rys:lstm_one_graph}
\end{figure}

Kolejną warstwą jest zamiana reprezentacji słowa na gęstą reprezentację wektorową opisaną w sekcji \ref{section:words_embeddings}. Do uzyskania tej reprezentacji użyto gotowe macierze udostępnione przez autorów metody na stronie internetowej\footnote{\url{https://nlp.stanford.edu/projects/glove/}}. Do wyboru były macierze przygotowane na różnych zbiorach danych oraz o różnych szerokościach wektorów. Testy przeprowadzono z użyciem zagłębień słów wytrenowanych na zbiorze danych pochodzących z Wikipedii oraz z archiwum danych tekstowych Gigaword 5 (ang. \textit{English Gigaword Fifth Edition}). Wybór szerokości wektorów podyktowany był złożonością modelu i czasem nauki, z pośród zbioru 50, 100, 200, 300 zdecydowano się na szerokość 100. Podsumowując złożoność tego etapu zawiera on prawie 2 miliony wewnętrznych parametrów, które są zamrożone na czas uczenia. Rysunek \ref{rys:lstm_one_table} przedstawia szczegóły budowy jednowarstwowej architektury LSTM, wraz z tą warstwą o nazwie \textit{Embedding}.

\begin{figure}[t]
\centering\includegraphics[width=10cm]{figures/reports/lstm_one_table.png}
\fcmfcaption{Tabela przedstawiająca szczegóły budowy jednowarstwowej architektury LSTM.}\label{rys:lstm_one_table}
\end{figure}

Do stworzenia kolejnej warstwy użyto gotową reprezentację LSTM z pakietu tensorflow o nazwie \textit{keras.layers.LSTM}, która udostępnia możliwość definiowania konkretnych szczegółów implementacyjnych. W warstwie tej użyto domyślną funkcję aktywacji tangens hiperboliczny. Po obserwacji krzywej uczenia reprezentującej dokładność oraz wartość funkcji straty zauważono zjawisko przeuczenia. Dokładność dla zbioru uczącego rosła wraz z malejącą dokładnością dla zbioru walidacyjnego. Z tego powodu zdecydowano się na użycie metody przerywania (ang. \textit{dropout}), zdefiniowanej w tej warstwie. Pomogło to zmniejszyć wpływ zjawiska przeuczania się modelu.

Ostatnim etapem jest wybór odpowiedniej klasy za pomocą warstwy jednokierunkowej. Wyjście z poprzedniej warstwy nazywane stanem ukrytym (ang. \textit{hidden state}) o szerokości 120 jest gęsto połączone z 4 neuronami decydującymi odpowiednio o każdej z kolejnych etykiet emocji. Jako funkcję aktywacji tej warstwy, zgodnie z sugestią organizatorów konkursu użyto funkcję sigmoidalną. Zauważona jednak została niekompatybilność tej funkcji z konkretnym zastosowaniem dla klasyfikacji wieloklasowej co zostało poprawione w kolejno zaprezentowanych modelach.  

Ostatnim elementem budowy tego modelu jest wybór algorytmu optymalizacyjnego. Wybrano algorytm RMSprop \cite{ruder2016overview} (ang. \textit{Root Mean Square Propagation}), który dobrze radzi sobie z wygasającymi wskaźnikami uczenia oraz przeciwdziała obliczeniowym problemom numerycznym. Jako funkcję straty użyto \textit{Categorical Cross-Entropy loss}, która bardzo dobrze radzi sobie w problemach klasyfikacji wielu klas.

\section{Głęboka architektura LSTM}

Głęboka architektura LSTM jest rozszerzeniem architektury jednowarstwowej opisanej w punkcie \ref{section:one_lstm}. Głównym celem było porównanie złożoności architektury płytkiej i głębokiej oraz wpływ rozszerzenia modelu o kolejne warstwy na wynik. Modyfikacje polegały przede wszystkim na dodaniu kolejnych warstw oraz lekkie modyfikacje parametrów oraz funkcji wewnątrz modelu. Pierwsze dwie warstwy, wejściowa (\textit{Input Layer}) oraz mapująca słowa na gęstą reprezentację wektorową (\textit{Embedding}) zostały bez zmian.

Do rozszerzenia jednowarstwowej części sieci rekurencyjnej zamiast zwykłych komórek LSTM użyto dwukierunkowych komórek LSTM. Z założenia rozszerzenie to powinno poprawić wydajność modelu przy problemach z klasyfikacją sekwencji \cite{ding2018densely}. W tym przypadku można było użyć tego rozszerzenia ponieważ od razu są dostępne wszystkie sekwencje czasowe sekwencji wejściowej. W tym momencie dwukierunkowe komórki LSTM łączą dwie ukryte warstwy o przeciwnych kierunkach. Dzięki tej formie uczenia warstwa wyjściowa może jednocześnie uzyskiwać informacje z przeszłych jak i przyszłych stanów, co nie było możliwe przy użyciu podstawowych komórek LSTM. Zabieg ten pomaga lepiej zrozumieć kontekst gdyż znaczenie danego słowa może zależeć także od słów, które są przed danym słowem. Dodatkowo zamiast jednej warstwy sieci rekurencyjnej użyto łącznie trzy warstwy używające dwukierunkowego LSTM. Pierwsze dwie warstwy na wyjściu oprócz ukrytego stanu zwracają także sekwencję o długości 200, którą przekazują na wejście kolejnej warstwy sieci LSTM. Widać to na rysunku \ref{rys:lstm_deep_graph}, który przedstawia graf głębokiej architektury LSTM.

\begin{figure}[t]
\centering\includegraphics[width=8cm]{figures/reports/lstm_deep_graph.png}
\fcmfcaption{Graf przedstawiający budowę głębokiej architektury LSTM.}\label{rys:lstm_deep_graph}
\end{figure}

Kolejną modyfikacją było dodanie kolejnych warstw sieci gęstej. Wcześniej użytą jedną warstwę zawierającą 4 neurony wyjściowe rozszerzono o kolejne dwie warstwy gęste. Pierwsza z nich zawiera 120 neuronów a druga zawiera 64 neurony. Szczegóły tej budowy przedstawiono na rysunku \ref{rys:lstm_deep_table}, który przestawia tabelę prezentującą szczegóły budowy głębokiej architektury LSTM. Do dodanych warstw użyto innej funkcji aktywacji jaką jest jednostronnie obcięta funkcja liniowa (ang. \textit{ReLU}), która stała się standardem do użycia w wewnętrznych warstwach gęstej sieci neuronowej \cite{xu2015empirical}. W ostatniej warstwie także zamieniono funkcję aktywacji na funkcję \textit{softmax}, która lepiej nadaje się do klasyfikacji wieloklasowej.

\begin{figure}[t]
\centering\includegraphics[width=10cm]{figures/reports/lstm_deep_table.png}
\fcmfcaption{Tabela przedstawiająca szczegóły budowy głębokiej architektury LSTM.}\label{rys:lstm_deep_table}
\end{figure}

Porównując dokonane rozszerzenia modelu z jedną warstwą LSTM, oraz wykonując opisane modyfikacje warstw zwiększyła się złożoność modelu. Prostym sposobem na porównanie modeli jest sprawdzenie liczby parametrów, które definiują zachowanie sztucznych neuronów, inaczej nazywane wagami neuronu. W tabeli \ref{tab:tabela_modele} przedstawione są liczby parametrów podzielonych na dwie grupy. Parametry trenowalne modelu to są wagi, które ulegają modyfikacji, liczba tych parametrów zwiększyła się około dziewięciokrotnie co ilustruję skalę zmian. Parametry stałe to są wagi użyte do generacji gęstych reprezentacji wektorowych, które były zamrożone na czas nauki modeli.  

\begin{table}[t]
\fcmtcaption{Tabela porównująca szczegóły budowy poszczególnych model.}\label{tab:tabela_modele}
\centering\footnotesize%
\begin{tabular}{c c c c}
\toprule
model & parametry trenowalne & parametry stałe & SUMA \\
\midrule
Jednowarstwowy LSTM   & 106,564 & 1,683,200 & 1,789,764 \\
Głęboki LSTM   & 942,204 & 1,683,200 & 2,625,404 \\
BERT todo   & xx & xx & xxx \\
\bottomrule
\end{tabular}
\end{table}

% \section{Głęboka architektura BERT}

% TODO \cite{devlin2018bert} BERT
\chapter{Ewaluacja modeli}

\section{Przebieg nauki}

\ref{rys:lstm_one_deep_comparison} Wykresy przedstawiające przebieg nauki dla modeli jednowarstwowego LSTM oraz głębokiego LSTM

\begin{figure}[t]
\centering\includegraphics[width=\textwidth]{figures/reports/lstm_one_deep_comparison.png}
\fcmfcaption{Wykresy przedstawiające przebieg nauki dla modeli jednowarstwowego LSTM oraz głębokiego LSTM}\label{rys:lstm_one_deep_comparison}
\end{figure}

\section{Metryki}

TODO

\section{Wyniki}

\begin{table}[t]
\fcmtcaption{Tabela pokazująca wyniki poszczególnych modeli.}\label{tab:tabela_results}
\centering\footnotesize%
\begin{tabular}{c c c c c}
\toprule
model & dokładność & mikro precyzja & micro czułość & mikro F1 \\
\midrule
Jednowarstwowy LSTM   & 0.85 & 0.49 & 0.70 & 0.58 \\
Głęboki LSTM   & 0.87 & 0.51 & 0.71 & 0.60 \\
BERT todo   & xx & xx & xxx & xxx \\
\bottomrule
\end{tabular}
\end{table}
\chapter{Podsumowanie}

Celem pracy było opracowanie modelu uczenia maszynowego opartego na głębokich sieciach neuronowych w celu detekcji emocji w dialogach. W jego ramach zostały zaprojektowane oraz przetestowane trzy różne architektury sieci neuronowych oraz została przeprowadzona analiza porównawcza tych modeli. Ponad to wszystkie zadania, które były niezbędne do zrealizowania tego celu także zostały wykonane. Są to między innymi zadania zapoznania się z literaturą dotyczącą głębokich sieci neuronowych i modeli emocji, przegląd oraz zapoznanie się z dostępnymi frameworkami do uczenia głębokiego, wstępna analiza wybranego zbioru danych oraz ewaluacja stworzonych modeli.

Rozpoznawanie emocji z tekstu jest zadaniem, które wymaga wielu czynników do poprawnego zrozumienia danego przekazu. Aby móc odkryć właściwą etykietę emocji należy spojrzeć na przekaz jako na całość, a nie na pojedyncze wyrazy, które mogą nieść inne znaczenie w różnych kontekstach. W podobny sposób objawia się działanie przedstawionych architektur do zrozumienia języka pisanego. Tradycyjne sieci neuronowe, proste metody uczenia maszynowego lub metody zliczające nie są w stanie odkryć znaczenia wypowiedzi jako całości. Bardzo ważnym elementem jest zrozumienie kontekstu oraz poradzenie sobie z problemami takimi jak występowanie sarkazmu lub ukrytego znaczenia.

Architektury korzystające z rekurencyjnych komórek LSTM są skonstruowane do rozumienia długotrwałych zależności i połączeń między wyrazami występującymi w zdaniu. Dzięki nim można było odkryć znaczenie przekazu, które zależało od kilku poprzednich wypowiedzi. Jeszcze lepsza okazała się architektura korzystająca z modelu BERT. Zastosowane w niej mechanizmy samoobserwacji oraz złożona budowa pozwoliła na jeszcze dokładniejsze zrozumienie tekstu.

Największym problemem w trakcie realizacji projektu okazały się ograniczenia zasobowe oraz długi czas przetwarzania modeli. Korzystanie z lokalnego laptopa do przeprowadzenia obliczeń oraz nauki modeli okazał się procesem zbyt długotrwałym oraz powodującym dużą zajętość zasobów takich jak pamięć oraz procesor. Pomocne okazały się serwisy udostępniające moce obliczeniowe w chmurze. Platforma Colab, użyta do trenowania modeli udostępniała dodatkowo dostęp do kart graficznych umożliwiających jeszcze efektywniejsze trenowanie modeli.

Do realizacji pracy użyto tylko niektórych z najbardziej popularnych metod głębokiego uczenia. W trakcie realizacji pracy wydano najnowszą architekturę określaną jako nowy stan techniki w przetwarzaniu języka naturalnego \textit{GPT-3} \cite{brown2020language}. Do uzyskania jeszcze lepszych wyników warto by przetestować działanie podobnych architektur oraz poświęcić więcej czasu na lepszym doborze parametrów. Istnieje także możliwość rozwijania obecnych architektur o inne zbiory danych oraz dostosowanie ich do użycia w praktyce w prawdziwych systemach informatycznych.

%--------------------------------------
% Literatura
%--------------------------------------

\bibliographystyle{plain}{\raggedright\sloppy\small\bibliography{bibliografia}}

%--------------------------------------
% Dodatki
%--------------------------------------

% EDIT: Te 3 linijki nie były zakomentowane, te nowe page może trzeba bedzie odkomentować
% \cleardoublepage\appendix%
% \newpage
% \chapter{Płyta CD}

Płyta CD z elektroniczną wersją pracy, projektem oraz danymi wejściowymi.

%--------------------------------------
% Informacja o prawach autorskich
%--------------------------------------

\ppcolophon

\end{document}
